\documentclass{article}
\usepackage[a4paper, total={7.5in, 10in}]{geometry}
\usepackage{pifont}
\usepackage{hyperref}
\usepackage{titling}
 

\hypersetup{
    colorlinks=true,
    linkcolor=blue,
    filecolor=magenta,      
    urlcolor=cyan,
    pdftitle={Overleaf Example},
    pdfpagemode=FullScreen,
    }

\urlstyle{same}

\begin{document}
\title{\vspace*{-4em}\textbf{JORDAN COX}}
\author{}
\date{}

\maketitle

\begin{center}
    \vspace*{-7em}
    (256)924-8347 \ding{71} \url{github.com/jordo47}
\end{center}

\subsection*{Technical Skills}
\subsubsection*{\hspace*{0.5cm} Programming Languages}
\hspace*{1cm} C, C++, C\verb|#|, Java, Python, hlsl, TypeScript, SQL, HTML, CSS, {\LaTeX}

\subsubsection*{\hspace*{0.5cm} Development Environments}
\hspace*{1cm} VS Code, Visual Studio, Spring Tool Suite

\subsubsection*{\hspace*{0.5cm} Tools \& Etc..}
\hspace*{1cm} Git, Subversion, Jira, Coverity, Microsoft Office, React, Node.js, Unity


\subsection*{Work Experience}

\subsection*{\hspace*{0.15cm} SMX Tech \hfill \textit{April 2022 - January 2023}}
\subsubsection*{\hspace*{0.15cm} Fullstack Software Engineer \hfill Huntsville, AL \textit{Remote}}

\begin{itemize}
    \item I worked on a team developing logistics software. I supported
     the legacy Java application and modern React application.
    \item For the legacy application, I would create new data types, add 
    them to the database, and perform maintenance to algorithms to utilize 
    the new data. Then I would represent this data in line graphs or tables.
    \item For the React application, I would build form components that
    would take user input data and I'd apply algorithms I would write in Java 
    to the data, then I would supply that data as a modular React table component.
    \item The team I worked on would have scrum meetings twice weekly, we used
    Git for version control, and we would utilize Jira for tracking our tasks, 
    testing, etc.
\end{itemize}

\subsection*{\hspace*{0.15cm} Leonardo DRS \hfill \textit{February 2020 - August 2021}}

\subsubsection*{\hspace*{0.15cm} Software Engineer \hfill Huntsville, AL \textit{Hybrid}}

\begin{itemize}
    \item I worked on multiple projects at DRS, including
     an Encryption/Validation API in C++, a Serial Downloader 
     C\verb|#| Windows Form application, and a Tank Diagnostics 
     application written in C++ which would lock the user out of 
     the operating system (Windows XP/7).
    \item I used tools and libraries like Git,
    Subversion, Jira, Coverity, VMWare, OpenSSL, CMake, GoogleTest, 
    DotRas, RASDial and SiteKiosk.
    \item I would setup the hardware to house our software, 
     which included installing the operating system, updating the BIOS,
     setting up the user accounts, and the applications. This sometimes
     included writing batch scripts.
    \item I also would work in SIL (Software-in-the-Loop) environments which
     would simulate an Abrams tank.
    \item I created and updated documentation for the projects I worked on
     using Microsoft Office (mostly Word, PowerPoint, and Visio)
    \item I was an ambassador of the company by interfacing with customers
     to present design decisions and overall progress on the implementation
     of software products.
    \item I also led a couple groups of interns on a project to create 
    an application and database to store weekly status updates of employees
    for our management during covid.
    
\end{itemize}

\subsection*{\hspace*{0.15cm} National Science Foundation \hfill \textit{May 2019 - August 2019}}

\subsubsection*{\hspace*{0.15cm} Undergraduate Researcher \hfill Auburn University}

\begin{itemize}
    \item I was awarded the Research Experiences for Undergraduates grant from 
    the National Science Foundation where I researched Cloud Computing infrastructure.
    \item I helped model different IoT Cloud Computing topologies ranging from large
     fog to edge computing environments in a software simulation using Java, and
     then graphed the results using MatLab.
    \item I also created a posterboard to present my research, and I helped
     write a research paper in {\LaTeX}
    
\end{itemize}

\subsection*{Personal Projects}
\subsubsection*{\hspace*{0.15cm} Procedural Mesh}

\begin{itemize}
    \item This project is one of several I've been making to teach myself 
     graphics programming using C\verb|#|, Unity, and High-Level Shader Language (hlsl).
    \item It procedurally generates a 3D mesh for different shapes, by taking
     an input resolution and creating a bunch of vertices based on the desired resolution.
     It then draws a bunch of small triangles between the vertices to create the
     shapes. 
    \item By tracking the position, normal, and tangent values of each vertex, I'm
     able to place a texture mapping over the shape, to give it a color and 3D texture. 
\end{itemize}


\subsection*{Education}
\subsubsection*{Auburn University \hfill \textit{August 2017 - December 2019}}
Bachelor of Engineering, Software \hfill GPA: 3.7 \textit{Magna Cum Laude}
\subsubsection*{Calhoun Community College \hfill August 2013 - May 2017}
Associate of Science, General Studies \hfill GPA: 3.75 \textit{Magna Cum Laude}



\end{document}